

%   COMP3310 UDP tutorial
%   Written by H Fisher, ANU, 2024
%   Creative Commons CC-BY-4.0 license

%   LaTeX markup for XeTeX with a standard texlive distribution
%   Command to build:
%       xelatex 01-udp.tex


%%%%    XeLaTeX common layout for COMP3310 tutorials

% Article class because tutorial instructions are short
\documentclass[a4paper,10pt,onecolumn,oneside,notitlepage]{extarticle}

% XeLaTex default input and output is UTF-8, so no need for character coding

\usepackage{changepage}
\usepackage{xspace}
\usepackage{graphicx}


%%  Fonts and characters

% With XeLaTeX can now use system fonts
\usepackage{fontspec}
\usepackage{unicode-math}

% Fonts! Everyone's favourite bike-shedding exercise.
% These fonts are distributed with texlive so should build everywhere.
\setmainfont{TeX Gyre Pagella}
\setsansfont{TeX Gyre Heros}
\setmonofont{TeX Gyre Cursor}
\setmathfont{TeX Gyre Pagella Math}


%%      Document structure

% Top of first page
\newcommand{\TITLE}[1] {%
        \begin{center}%
        \sffamily\bfseries\large%
        COMP3310/etc \DSH #1%
        \normalsize
        \end{center}%
}

% Section, subsection are sans-serif bold at normal text size,
% less spacing than usual between heading and paragraph.
\usepackage[sf,bf,tiny,raggedright,compact]{titlesec}

% Numbered subheading
\newcommand{\SECTION}[1] {%
        \section{#1}%
}

% Minor subheading
\newcommand{\MINOR}[1] {%
        \subsection*{#1}%
}


%%  Body text

% Try not to hyphenate
\sloppy

% Don't indent first line, put space between paragraphs
\setlength{\parindent}{0pt}
\setlength{\parskip}{1ex plus1ex minus0.2ex}

% Block of program code or console input and output, indented.
% Because code often has backslashes and curly brackets, always
% follow \begin{CODE} with \begin{verbatim} and finish with
% \end{verbatim} before \end{CODE}
% (No you can't put the begin/end verbatim in the environment.)
\newenvironment{CODE}
        {\begin{adjustwidth}{2em}{2em}\begin{ttfamily}\small}
        {\end{ttfamily}\end{adjustwidth}}

% One liner code/console command
\newcommand{\CODELINE}[1] {%
        \begin{CODE}%
        #1
        \end{CODE}%
}

% Text indented and italic
\newenvironment{IMPORTANT}
        {\begin{adjustwidth}{2em}{2em}\begin{itshape}}
        {\end{itshape}\end{adjustwidth}}


%%  Shortcuts

\newcommand{\COPYRIGHT} {%
        \small%
        \centerline{\rule{0.5\linewidth}{0.5pt}} %
        Written by H Fisher, Australian National University, 2024 \\
        Creative Commons BY-NC-SA license%
        \normalsize
}

% With smart editors interfering, safer not to type quotes directly
\newcommand{\DQ}[1] {%
        \textquotedblleft{}#1\textquotedblright{}%
}

\newcommand{\SQ}[1] {%
        \textquoteleft{}#1\textquoteright{}%
}

% Names of computers, hosts, programs
\newcommand{\NAME}[1] {%
        \texttt{\textbf{#1}}%
}

% For text, true dash to stop typographers complaining
\newcommand{\DSH} {\textendash{}\xspace}

% Bullet mark used outside list
\newcommand{\DOT} {\bullet\xspace}

% Introduce a new step in tutorial exercise
\newcounter{tutestep}
\newcommand{\STEP} {%
        \addtocounter{tutestep}{1}%
        \thetutestep.\xspace%
}


\begin{document}

\TITLE{UDP Client Server}


\MINOR{Outline}

In this tutorial you will

\DOT Compile and run a pair of programs implementing a client-server design

\DOT Begin studying a few of the problems that can occur in Internet programs

\DOT Discover the network addresses of computers and understand two special values

\DOT Study how text is sent and received over the Internet


\SECTION{Client Server}

For this tutorial you will need two command line shells, one for running
the server (\NAME{udpServer.py} or \NAME{UdpServer.java}) and one for the client
(\NAME{udpClient.py} or \NAME{UdpClient.java}).

Have a quick look through the code, but don't worry about understanding every
line just yet. For now, just identify the main client or server loop and the two
functions or methods that the loop relies on.

Ideally you should have WireShark running as well, so you can see the network
packets being sent back and forth. Initially you just need to watch UDP packets
on the 127.0.0.1 (loopback) interface.

Run the server program in one shell. Nothing should happen: the server is \SQ{passive}
and waits for clients to connect.

Run the client program in another shell, and type a line. This is the \NAME{request}
sent, which should generate output in both windows.
The server generates some log messages to show what it is doing, which don't affect
the client but are useful for debugging. The client also has some log messages, and
prints the \NAME{reply} or \NAME{response} it gets from the server.

If you have WireShark running, you should see a packet being sent in each direction.

The client program stops on terminal EOF. (Control-D or Control-Z depending on your
system.) Do this. What happens in the server window? In WireShark?

Run the client again, and type a few more words.

Open a third shell, and run a second client program, so you have two connected to the
same server. Try typing different words in each client shell.

\begin{IMPORTANT}
If you study the logging messages in the server window, you should see a difference
between those for the first run of the client program and those from the second.
This will be explained in detail later.
\end{IMPORTANT}

The server program does not read from input, so won't respond to EOF. Instead use
Control-C to stop it.


\SECTION{Reliability}

This pair of programs are using UDP, \NAME{Unreliable Data Protocol}. Since real
network errors are very rare when both programs are running on the same computer,
the server program has been modified to make it misbehave on two special requests,
\SQ{it} and \SQ{ni}. Read the server program code and find what each of these will
do within \NAME{serverLoop}.

\begin{IMPORTANT}
For the next steps, try to predict what will happen before you actually type in any
requests. If your prediction is right, good. If your prediction is wrong but you
can understand why, that's good too.
\end{IMPORTANT}

Make sure you have a server and client running in separate shells.

1. Type just \texttt{it} as the request. Wait several seconds. (Read the client program
code to find out exactly how long.)

2. Type three requests (Enter after each) \texttt{hello}, \texttt{ni}, \texttt{world}.
What happens?

Type in three or four more words.

EOF the client and run it again \DSH this is the easiest way to escape when
a client and server are not synchronised.

3. Control-C the \emph{server} program. Wait a few seconds. What happens to the client?

Type a request in the client. What happens? What do you see in Wireshark?

Still without a server, run the client program. Does it crash immediately?


\SECTION{Networking}

For this part of the exercise you will need to work with another student, with both your
computers connected to the same network. This could be two PCs in the same CSIT lab, or
two laptops connected to the same wireless network.

In a shell, type command \NAME{ifconfig} on a Linux or Mac, \NAME{ipconfig} on MS Windows.

This command shows the network interfaces on your computer and a lot of technical details
about each. For now the only information we need is the \texttt{inet} (Internet Address)
or \texttt{IPv4 Address}. This is four decimal numbers separated by periods \DSH the same
format as the \texttt{127.0.0.1} address shown in the server and client log messages.

\begin{IMPORTANT}
Once Upon A Time a computer would have only one or two network interfaces. Modern 21st
century computers may have lots of different network interfaces to search through.

You may or may not see different addresses in \emph{IPv6} format as well. Some future version
of this course will use IPv6 only, but not yet.
\end{IMPORTANT}

Both addresses should start with the same one or two numbers, indicating a shared network.
For example my two home computers have IPv4 addresses \texttt{10.1.1.113}
and \texttt{10.1.1.217}.

(It is \emph{possible} that this tutorial will run with two computers anywhere in the world,
since that is why we use the Internet, but there are a lot of potential problems
with no easy solutions. Both computers on the same network is more likely to work.)

Pick one computer to run the server and one for the client. The client will need the IP
address of the computer running the server. First verify that the client computer can
connect over the network with the \NAME{ping} command:

\begin{CODE}\begin{verbatim}
/Users/hugh/Desktop% ping 10.1.1.113
PING 10.1.1.113 (10.1.1.113): 56 data bytes
64 bytes from 10.1.1.113: icmp_seq=0 ttl=64 time=0.756 ms
64 bytes from 10.1.1.113: icmp_seq=1 ttl=64 time=0.363 ms
^C
\end{verbatim}\end{CODE}

1. Run the server program on the \SQ{server} computer (in the example above, the computer
with IP address \texttt{10.1.1.113}) with no command line arguments.
Run the client program on the other \SQ{client} computer with the IP address of the server
as the command line argument:
\CODELINE{python udpClient.py 10.1.1.113}
or
\CODELINE{java UdpClient 10.1.1.113}

When you type a request, the client will crash! But if you run a client on the server computer
itself, it will be able to send a request and get a reply. This is because the server IP address
\texttt{127.0.0.1} only works for programs on the same computer.

2. Control-C the server and restart with the IP address of the computer it is running on:
\CODELINE{python udpServer.py 10.1.1.217}
or
\CODELINE{java UdpServer 10.1.1.217}

Run the client again on the other computer, with the server IP address, and this time
it should work.

3. Control-C the server and start it again with this special address:
\CODELINE{python udpServer.py 0.0.0.0}
or
\CODELINE{java UdpServer 0.0.0.0}

Make sure that the client is still working.

\begin{IMPORTANT}
\texttt{127.0.0.1} means \SQ{inside this computer}.

\texttt{0.0.0.0} for a server means \SQ{Whatever interface this computer has}. It allows
the server program to be moved from computer to computer without needing to change the
IP address.
\end{IMPORTANT}

4. Swap the computer roles, with the server running on the former client computer and
the client on what used to be the server. If they don't work, check the IP addreses
you are using as command line arguments.


\SECTION{Messages}

The client and server programs both have a limit on the maximum size data that
will be read from a socket. Read the code to find out what these values are.

This tutorial comes with a file \texttt{words.txt} for testing the limits. Open this
file in a text editor or word processor: if you need to specify the encoding, UTF-8.
The file contains some French and Mandarin characters: if these are not displayed
properly on your system, instead appearing as small boxes, you may need to change
editor or terminal settings, or try a different editor.

Run a server and a client. (For this section they don't need to be on different
computers.)

From the editor with \texttt{words.txt} try copying \texttt{Test} and pasting into
the client terminal. Press Enter to send the request as normal.

1. Copy the long request (\SQ{extraterrestrial}) into the client. How many characters
are you sending?

How long is the reply from the server? Is this greater than the \texttt{MSG\_SIZE}
limit inside the server program? Read the server code: how can this happen?

2. Copy and send the very long request. Now what happens, and why?

\begin{IMPORTANT}
Ideally the maximum size of each request and reply should be part of the \NAME{protocol
specification}, so everyone agrees on what the limits are and can write programs that
interoperate. The programs may \NAME{negotiate} settings: \DQ{Can I send you 65535
bytes?} \DQ{No, only 16384}.

This tutorial is unfortunately more realistic: client and server programmers pick
numbers that seem about right, and only discover problems when strange thngs start
happening.
\end{IMPORTANT}

3. Copy and send the 16 characters in French, and the 10 characters in Mandarin.
What happens?

\begin{IMPORTANT}
Characters are not bytes. Older programming languages (and older programmers) assumed
that one byte would be enough for a character, or maybe two. Both are wrong and you
should allow up to 4 bytes (32 bits) per character.
\end{IMPORTANT}

4. Control-C the server. Increase the server limit in the program code and run it
again. Repeat until all the requests can be received without being truncated.


\SECTION{Optional}

Modify the client and server programs to use a shared library that contains the
maximum message size instead of having a separate limit within each program.

Modify the server program to shut down nicely on Control-C. (ie it should print out
\DQ{Done.} in the log messages.)

Change the client \NAME{sendRequest} to send a UTF-16 encoded string instead of the
current UTF-8. \emph{Don't} change anything else. What happens to the request and
reply?



\end{document}
