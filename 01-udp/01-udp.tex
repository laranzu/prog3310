

%   COMP3310 UDP tutorial
%   Written by Hugh Fisher, ANU, 2024-
%   Creative Commons CC-BY-NC-SA-4.0 license

%   LaTeX markup for XeTeX with a standard texlive distribution
%   Command to build:
%       xelatex 01-udp.tex


%%%%    XeLaTeX common layout for COMP3310 tutorials

% Article class because tutorial instructions are short
\documentclass[a4paper,10pt,onecolumn,oneside,notitlepage]{extarticle}

% XeLaTex default input and output is UTF-8, so no need for character coding

\usepackage{changepage}
\usepackage{xspace}
\usepackage{graphicx}


%%  Fonts and characters

% With XeLaTeX can now use system fonts
\usepackage{fontspec}
\usepackage{unicode-math}

% Fonts! Everyone's favourite bike-shedding exercise.
% These fonts are distributed with texlive so should build everywhere.
\setmainfont{TeX Gyre Pagella}
\setsansfont{TeX Gyre Heros}
\setmonofont{TeX Gyre Cursor}
\setmathfont{TeX Gyre Pagella Math}


%%      Document structure

% Top of first page
\newcommand{\TITLE}[1] {%
        \begin{center}%
        \sffamily\bfseries\large%
        COMP3310/etc \DSH #1%
        \normalsize
        \end{center}%
}

% Section, subsection are sans-serif bold at normal text size,
% less spacing than usual between heading and paragraph.
\usepackage[sf,bf,tiny,raggedright,compact]{titlesec}

% Numbered subheading
\newcommand{\SECTION}[1] {%
        \section{#1}%
}

% Minor subheading
\newcommand{\MINOR}[1] {%
        \subsection*{#1}%
}


%%  Body text

% Try not to hyphenate
\sloppy

% Don't indent first line, put space between paragraphs
\setlength{\parindent}{0pt}
\setlength{\parskip}{1ex plus1ex minus0.2ex}

% Block of program code or console input and output, indented.
% Because code often has backslashes and curly brackets, always
% follow \begin{CODE} with \begin{verbatim} and finish with
% \end{verbatim} before \end{CODE}
% (No you can't put the begin/end verbatim in the environment.)
\newenvironment{CODE}
        {\begin{adjustwidth}{2em}{2em}\begin{ttfamily}\small}
        {\end{ttfamily}\end{adjustwidth}}

% One liner code/console command
\newcommand{\CODELINE}[1] {%
        \begin{CODE}%
        #1
        \end{CODE}%
}

% Text indented and italic
\newenvironment{IMPORTANT}
        {\begin{adjustwidth}{2em}{2em}\begin{itshape}}
        {\end{itshape}\end{adjustwidth}}


%%  Shortcuts

\newcommand{\COPYRIGHT} {%
        \small%
        \centerline{\rule{0.5\linewidth}{0.5pt}} %
        Written by H Fisher, Australian National University, 2024 \\
        Creative Commons BY-NC-SA license%
        \normalsize
}

% With smart editors interfering, safer not to type quotes directly
\newcommand{\DQ}[1] {%
        \textquotedblleft{}#1\textquotedblright{}%
}

\newcommand{\SQ}[1] {%
        \textquoteleft{}#1\textquoteright{}%
}

% Names of computers, hosts, programs
\newcommand{\NAME}[1] {%
        \texttt{\textbf{#1}}%
}

% For text, true dash to stop typographers complaining
\newcommand{\DSH} {\textendash{}\xspace}

% Bullet mark used outside list
\newcommand{\DOT} {\bullet\xspace}

% Introduce a new step in tutorial exercise
\newcounter{tutestep}
\newcommand{\STEP} {%
        \addtocounter{tutestep}{1}%
        \thetutestep.\xspace%
}


\begin{document}

\TITLE{UDP Client Server}


\MINOR{Outline}

In this tutorial you will

\DOT Compile and run a pair of programs implementing a client-server design

\DOT Begin studying a few of the problems that can occur in Internet programs

\DOT See how text is sent and received over the Internet

\begin{IMPORTANT}
This tutorial assumes you have completed the previous \NAME{Programming Warm Up}.

There is a lot to do in this tutorial. Don't worry if you cannot finish everything
within the time slot, but do allocate time if necessary to finish. Programming
tutorials are not marked, but they are practice for the assignments.
\end{IMPORTANT}


\SECTION{Client Server}

For this tutorial you will need two command line shells, one for running
the server (\NAME{udpServer.py} or \NAME{UdpServer.java}) and one for the client
(\NAME{udpClient.py} or \NAME{UdpClient.java}).

Have a quick look through the code, but don't worry about understanding every
line just yet. For now, just identify the main client or server loop and the two
functions or methods that the loop relies on.

Both the client and server programs have optional command line arguments. For
this tutorial we will not use these.

Ideally you should have WireShark running as well, so you can see the network
packets being sent back and forth. You just need to watch UDP packets
on the 127.0.0.1 (loopback) interface.

Run the server program in one shell. Nothing should happen: the server is \SQ{passive}
and waits for clients to connect.

Run the client program in another shell, and type a line. This is the \NAME{request}
sent, which should generate output in both windows.
The server generates some log messages to show what it is doing, which don't affect
the client but are useful for debugging. The client also has some log messages, and
prints the \NAME{reply} or \NAME{response} it gets from the server.

In WireShark you should see a packet being sent in each direction.

The client program stops on terminal EOF. (Control-D or Control-Z depending on your
system.) Do this. What happens in the server window? In WireShark?

Run the client again, and type a few more words.

Open a third shell, and run a second client program, so you have two connected to the
same server. Try typing different words in each client shell.

\begin{IMPORTANT}
If you study the logging messages in the server window, you should see a difference
between those for the first run of the client program and those from the second.
This will be explained later in the course.
\end{IMPORTANT}

The server program does not read from input, so won't respond to EOF. Instead use
Control-C to stop it. Try this \emph{while} a client is still running. What happens
to the client? What do you see in WireShark?

EOF all your clients, then run a client without any server. What happens and when?


\SECTION{Reliability}

This pair of programs are using UDP, \NAME{Unreliable Data Protocol}. Since real
network errors are very rare when both programs are running on the same computer,
we will modify the server program to make it misbehave on special requests,
\DQ{it} and \DQ{ni}. If you're not sure how to start, the \NAME{knight} program
from the previous tutorial has code you could use.

\STEP Modify the server program so that if the request is \DQ{it}, the server does
not reply. (But it should print a log message to the terminal saying that it has
received a special message.)

\begin{IMPORTANT}
For the next steps, try to predict what will happen before you actually type in any
requests. If your prediction is right, good. If your prediction is wrong but you
can understand why, that's good too.
\end{IMPORTANT}

Run the modified server, then a client. Type just \texttt{it} as the request. Wait
several seconds (read the client program code to find out exactly how long).

\STEP Modify the server program so that if the request is \DQ{ni} it sends three replies,
not one.

Type a few requests (Enter after each) \texttt{hello}, \texttt{ni}, \texttt{world}.
What happens?

Type in three or four more words.


\SECTION{Messages}

The client and server programs both have a limit on the maximum size data that
will be read from a socket. Read the code to find out what these values are.

This tutorial comes with a file \texttt{words.txt} for testing the limits. Open this
file in a text editor or word processor: if you need to specify the encoding, UTF-8.
The file contains some French and Mandarin characters: if these are not displayed
properly on your system, instead appearing as small boxes, you may need to change
editor or terminal settings, or try a different editor.

Start a server and a client.

From the editor with \texttt{words.txt} try copying \texttt{Test} and pasting into
the client terminal. Press Enter to send the request as normal.

\STEP Copy the long request (\SQ{extraterrestrial}) into the client. How many characters
are you sending?

How long is the reply from the server? Is this greater than the \texttt{MSG\_SIZE}
limit inside the server program? Read the server code: how can this happen?

\STEP Copy and send the very long request. Now what happens, and why?

\begin{IMPORTANT}
The maximum size of each request and reply should be part of the \NAME{protocol
specification}, so everyone agrees on what the limits are and can write programs that
interoperate. Or the programs may \NAME{negotiate} settings: \SQ{Can I send you 65535
bytes?} \SQ{No, only 16384}.

This tutorial shows what often happens in the real world: client and server
programmers pick numbers that seem about right, and only discover problems when
strange things start happening. Don't do this.
\end{IMPORTANT}

\STEP Copy and send the 16 characters in French, and the 10 characters in Mandarin.
What happens?

\begin{IMPORTANT}
Characters are not bytes and strings are complicated.
\end{IMPORTANT}

\STEP Control-C the server. Increase the server limit in the program code and run it
again. Repeat until all the requests can be received without being truncated.


\SECTION{More Coding}

Change the client \NAME{sendRequest} to send a UTF-16 encoded string instead of the
current UTF-8. \emph{Don't} change anything else. In Wireshark, what happens to the
request and reply? Change back to UTF-8 afterwards.

Modify the client and server programs to use a shared package that contains the
maximum message size instead of having a separate limit within each program.

\emph{Optional}: Have a single pair of send and receive functions/methods
in the shared package that are used by both the client and server.

\emph{Optional}: Modify the server program to shut down nicely on Control-C:
it should close the socket and print something as a log messages.


\COPYRIGHT

\end{document}
