

%   COMP3310 IP address tutorial
%   Written by Hugh Fisher, ANU, 2025
%   Creative Commons CC-BY-NC-SA-4.0 license

%   LaTeX markup for XeTeX with a standard texlive distribution
%   Command to build:
%       xelatex 03-ipaddr.tex


%%%%    XeLaTeX common layout for COMP3310 tutorials

% Article class because tutorial instructions are short
\documentclass[a4paper,10pt,onecolumn,oneside,notitlepage]{extarticle}

% XeLaTex default input and output is UTF-8, so no need for character coding

\usepackage{changepage}
\usepackage{xspace}
\usepackage{graphicx}


%%  Fonts and characters

% With XeLaTeX can now use system fonts
\usepackage{fontspec}
\usepackage{unicode-math}

% Fonts! Everyone's favourite bike-shedding exercise.
% These fonts are distributed with texlive so should build everywhere.
\setmainfont{TeX Gyre Pagella}
\setsansfont{TeX Gyre Heros}
\setmonofont{TeX Gyre Cursor}
\setmathfont{TeX Gyre Pagella Math}


%%      Document structure

% Top of first page
\newcommand{\TITLE}[1] {%
        \begin{center}%
        \sffamily\bfseries\large%
        COMP3310/etc \DSH #1%
        \normalsize
        \end{center}%
}

% Section, subsection are sans-serif bold at normal text size,
% less spacing than usual between heading and paragraph.
\usepackage[sf,bf,tiny,raggedright,compact]{titlesec}

% Numbered subheading
\newcommand{\SECTION}[1] {%
        \section{#1}%
}

% Minor subheading
\newcommand{\MINOR}[1] {%
        \subsection*{#1}%
}


%%  Body text

% Try not to hyphenate
\sloppy

% Don't indent first line, put space between paragraphs
\setlength{\parindent}{0pt}
\setlength{\parskip}{1ex plus1ex minus0.2ex}

% Block of program code or console input and output, indented.
% Because code often has backslashes and curly brackets, always
% follow \begin{CODE} with \begin{verbatim} and finish with
% \end{verbatim} before \end{CODE}
% (No you can't put the begin/end verbatim in the environment.)
\newenvironment{CODE}
        {\begin{adjustwidth}{2em}{2em}\begin{ttfamily}\small}
        {\end{ttfamily}\end{adjustwidth}}

% One liner code/console command
\newcommand{\CODELINE}[1] {%
        \begin{CODE}%
        #1
        \end{CODE}%
}

% Text indented and italic
\newenvironment{IMPORTANT}
        {\begin{adjustwidth}{2em}{2em}\begin{itshape}}
        {\end{itshape}\end{adjustwidth}}


%%  Shortcuts

\newcommand{\COPYRIGHT} {%
        \small%
        \centerline{\rule{0.5\linewidth}{0.5pt}} %
        Written by Hugh Fisher, Australian National University, 2024 \\
        Creative Commons BY-NC-SA license%
        \normalsize
}

% With smart editors interfering, safer not to type quotes directly
\newcommand{\DQ}[1] {%
        \textquotedblleft{}#1\textquotedblright{}%
}

\newcommand{\SQ}[1] {%
        \textquoteleft{}#1\textquoteright{}%
}

% Names of computers, hosts, programs
\newcommand{\NAME}[1] {%
        \texttt{\textbf{#1}}%
}

% For text, true dash to stop typographers complaining
\newcommand{\DSH} {\textendash{}\xspace}

% Bullet mark used outside list
\newcommand{\DOT} {\bullet\xspace}

% Introduce a new step in tutorial exercise
\newcounter{tutestep}
\newcommand{\STEP} {%
        \addtocounter{tutestep}{1}%
        \thetutestep.\xspace%
}


\begin{document}

\TITLE{IP Addressing}


\MINOR{Outline}

In this tutorial you will

\DOT Discover the network addresses of computers and understand two special values

\DOT Study how port numbers can be used for transport layer multiplexing

This tutorial uses a pair of TCP client and server programs, which are the same as
those for \NAME{TCP Client Server} tutorial.

\SECTION{Network address}

For this part of the exercise you will need to work with another student, with both your
computers connected to the same network. This could be two PCs in the same CSIT lab, or
two laptops connected to the same wireless network.

\STEP In a shell on each computer, type \NAME{ifconfig} on a Linux or Mac,
\NAME{ipconfig} on MS Windows.

This command shows the network interfaces on your computer and a lot of technical details
about each. For now the only information we need is the \texttt{inet} (Internet Address)
or \texttt{IPv4 Address}. This is four decimal numbers separated by periods \DSH the same
format as the \texttt{127.0.0.1} address shown in the server and client log messages.

\begin{IMPORTANT}
Once Upon A Time a computer would have only one or two network interfaces. Modern 21st
century computers may have lots of different network interfaces to search through.

You may or may not see different addresses in \emph{IPv6} format as well. Some future version
of this course will use IPv6 only, but not yet.
\end{IMPORTANT}

Both addresses should start with the same first one to three numbers, indicating a
shared network.
For the examples my two home computers have IPv4 addresses \texttt{10.1.1.113}
and \texttt{10.1.1.217}. You will need to replace these with the addresses you found.

(It is \emph{possible} that this tutorial will run with two computers anywhere in the world,
since that is why we use the Internet, but there are a lot of potential problems
with no easy solutions. Both computers on the same network is more likely to work.)

Pick one computer to run the server and one for the client. The client will need the IP
address of the computer running the server, here \texttt{10.1.1.113}.
First verify that the client computer can
connect to the server computer over the network with the \NAME{ping} command:

\begin{CODE}\begin{verbatim}
/Users/hugh/Desktop% ping 10.1.1.113
PING 10.1.1.113 (10.1.1.113): 56 data bytes
64 bytes from 10.1.1.113: icmp_seq=0 ttl=64 time=0.756 ms
64 bytes from 10.1.1.113: icmp_seq=1 ttl=64 time=0.363 ms
^C
\end{verbatim}\end{CODE}

If the client cannot ping the server, repeat the \NAME{ifconfig} or \NAME{ipconfig} on
the server computer. You may have mistyped the address, or need to choose a different address.

\STEP Run the server program on the server computer (in the example above, the computer
with IP address \texttt{10.1.1.113}) with no command line arguments.
\CODELINE{python tcpServer.py}
or
\CODELINE{java TcpServer}

Run the client program on the other computer with the IP address of the server
as the command line argument:
\CODELINE{python tcpClient.py 10.1.1.113}
or
\CODELINE{java TcpClient 10.1.1.113}

The client will crash with some kind of connection exception. But if you open a second
shell on the server computer and run a client there, it will be able to send a request
and get a reply.
This is because the default server IP address \texttt{127.0.0.1}
only works for programs on the same computer.

\STEP Control-C the server and restart with the IP address of the computer it is running on:
\CODELINE{python tcpServer.py 10.1.1.113}
or
\CODELINE{java TcpServer 10.1.1.113}

Run the client again on the other computer, with the server IP address, and this time
it should work.

\STEP Control-C the server and start it again with this special address:
\CODELINE{python tcpServer.py 0.0.0.0}
or
\CODELINE{java TcpServer 0.0.0.0}

Run the client again: it should still work.

\begin{IMPORTANT}
\texttt{127.0.0.1} means \SQ{inside this computer}.

\texttt{0.0.0.0} for a server means \SQ{Whatever interface this computer has}. It allows
the server program to be moved from computer to computer without needing to change the
IP address.
\end{IMPORTANT}

\STEP Swap the computer roles, with the server running on the former client computer and
the client on what used to be the server. If they don't work, check the IP addresses
you are using as command line arguments.



\SECTION{Experiments}

\emph{Optional}: Try using an IPv6 address for server and client.


\COPYRIGHT

\end{document}
