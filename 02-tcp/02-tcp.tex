
%   COMP3310 TCP tutorial
%   Written by H Fisher, ANU, 2024
%   Creative Commons CC-BY-4.0 license

%   LaTeX markup for XeTeX with a standard texlive distribution
%   Command to build:
%       xelatex 02-tcp.tex


%%%%    XeLaTeX common layout for COMP3310 tutorials

% Article class because tutorial instructions are short
\documentclass[a4paper,10pt,onecolumn,oneside,notitlepage]{extarticle}

% XeLaTex default input and output is UTF-8, so no need for character coding

\usepackage{changepage}
\usepackage{xspace}
\usepackage{graphicx}


%%  Fonts and characters

% With XeLaTeX can now use system fonts
\usepackage{fontspec}
\usepackage{unicode-math}

% Fonts! Everyone's favourite bike-shedding exercise.
% These fonts are distributed with texlive so should build everywhere.
\setmainfont{TeX Gyre Pagella}
\setsansfont{TeX Gyre Heros}
\setmonofont{TeX Gyre Cursor}
\setmathfont{TeX Gyre Pagella Math}


%%      Document structure

% Top of first page
\newcommand{\TITLE}[1] {%
        \begin{center}%
        \sffamily\bfseries\large%
        COMP3310/etc \DSH #1%
        \normalsize
        \end{center}%
}

% Section, subsection are sans-serif bold at normal text size,
% less spacing than usual between heading and paragraph.
\usepackage[sf,bf,tiny,raggedright,compact]{titlesec}

% Numbered subheading
\newcommand{\SECTION}[1] {%
        \section{#1}%
}

% Minor subheading
\newcommand{\MINOR}[1] {%
        \subsection*{#1}%
}


%%  Body text

% Try not to hyphenate
\sloppy

% Don't indent first line, put space between paragraphs
\setlength{\parindent}{0pt}
\setlength{\parskip}{1ex plus1ex minus0.2ex}

% Block of program code or console input and output, indented.
% Because code often has backslashes and curly brackets, always
% follow \begin{CODE} with \begin{verbatim} and finish with
% \end{verbatim} before \end{CODE}
% (No you can't put the begin/end verbatim in the environment.)
\newenvironment{CODE}
        {\begin{adjustwidth}{2em}{2em}\begin{ttfamily}\small}
        {\end{ttfamily}\end{adjustwidth}}

% One liner code/console command
\newcommand{\CODELINE}[1] {%
        \begin{CODE}%
        #1
        \end{CODE}%
}

% Text indented and italic
\newenvironment{IMPORTANT}
        {\begin{adjustwidth}{2em}{2em}\begin{itshape}}
        {\end{itshape}\end{adjustwidth}}


%%  Shortcuts

\newcommand{\COPYRIGHT} {%
        \small%
        \centerline{\rule{0.5\linewidth}{0.5pt}} %
        Written by H Fisher, Australian National University, 2024 \\
        Creative Commons BY-NC-SA license%
        \normalsize
}

% With smart editors interfering, safer not to type quotes directly
\newcommand{\DQ}[1] {%
        \textquotedblleft{}#1\textquotedblright{}%
}

\newcommand{\SQ}[1] {%
        \textquoteleft{}#1\textquoteright{}%
}

% Names of computers, hosts, programs
\newcommand{\NAME}[1] {%
        \texttt{\textbf{#1}}%
}

% For text, true dash to stop typographers complaining
\newcommand{\DSH} {\textendash{}\xspace}

% Bullet mark used outside list
\newcommand{\DOT} {\bullet\xspace}

% Introduce a new step in tutorial exercise
\newcounter{tutestep}
\newcommand{\STEP} {%
        \addtocounter{tutestep}{1}%
        \thetutestep.\xspace%
}


\begin{document}

\TITLE{TCP Client Server}


\MINOR{Outline}

In this tutorial you will

\DOT Study the differences between UDP and TCP program design

\DOT Study how data is transmitted and received over TCP


This tutorial is very similar to the previous UDP client-server, so you should
know what to do. Before studying anything in detail make sure everything is
working. Run the server in one window, the client in another. Whatever
you type in the client window will be sent to the server and echoed back. EOF
in the client window to stop, Control-C for the server.

For this tutorial you should have WireShark (or \texttt{tcpdump}) running
on the loopback interface, port 3310.


\SECTION{Program Code}

1. The TCP client and server both use a shared package for reading and writing
lines of text, \NAME{sockLine.py} or \NAME{SockLine.java}. Have a look at the
code.

Sending a line is easy: append a newline special character and then encode
into UTF-8. It is only a couple of lines, but using a library means we cannot
forget either step.

Reading a line is very different. The UDP programs received packets, and each
package was a single request or response. TCP hides the packets, so a TCP
socket behaves like a sequential file of bytes. The library code reads one
byte at a time, but this does not mean that every byte is a packet. WireShark
will show you what is actually being sent.

2. The TCP server code is structured very differently. A UDP server responds
to client packets and does not really need any state. A TCP server receives
connection requests from a client, and each connection is a \NAME{session}
dedicated to that client.

The TCP server now has two loops, one for clients and one for requests by a
single client.


\SECTION{Optional: other programs}

The TCP server can echo the input from other programs, not just the client.
Run the server and try \NAME{telnet} or \NAME{nc} as the client. Or you can open
a web browser and type \texttt{localhost:3310} as the address to open.


\COPYRIGHT

\end{document}
