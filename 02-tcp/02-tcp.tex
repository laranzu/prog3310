
%   COMP3310 TCP tutorial
%   Written by Hugh Fisher, ANU, 2024
%   Creative Commons CC-BY-4.0 license

%   LaTeX markup for XeTeX with a standard texlive distribution
%   Command to build:
%       xelatex 02-tcp.tex


%%%%    XeLaTeX common layout for COMP3310 tutorials

% Article class because tutorial instructions are short
\documentclass[a4paper,10pt,onecolumn,oneside,notitlepage]{extarticle}

% XeLaTex default input and output is UTF-8, so no need for character coding

\usepackage{changepage}
\usepackage{xspace}
\usepackage{graphicx}


%%  Fonts and characters

% With XeLaTeX can now use system fonts
\usepackage{fontspec}
\usepackage{unicode-math}

% Fonts! Everyone's favourite bike-shedding exercise.
% These fonts are distributed with texlive so should build everywhere.
\setmainfont{TeX Gyre Pagella}
\setsansfont{TeX Gyre Heros}
\setmonofont{TeX Gyre Cursor}
\setmathfont{TeX Gyre Pagella Math}


%%      Document structure

% Top of first page
\newcommand{\TITLE}[1] {%
        \begin{center}%
        \sffamily\bfseries\large%
        COMP3310/etc \DSH #1%
        \normalsize
        \end{center}%
}

% Section, subsection are sans-serif bold at normal text size,
% less spacing than usual between heading and paragraph.
\usepackage[sf,bf,tiny,raggedright,compact]{titlesec}

% Numbered subheading
\newcommand{\SECTION}[1] {%
        \section{#1}%
}

% Minor subheading
\newcommand{\MINOR}[1] {%
        \subsection*{#1}%
}


%%  Body text

% Try not to hyphenate
\sloppy

% Don't indent first line, put space between paragraphs
\setlength{\parindent}{0pt}
\setlength{\parskip}{1ex plus1ex minus0.2ex}

% Block of program code or console input and output, indented.
% Because code often has backslashes and curly brackets, always
% follow \begin{CODE} with \begin{verbatim} and finish with
% \end{verbatim} before \end{CODE}
% (No you can't put the begin/end verbatim in the environment.)
\newenvironment{CODE}
        {\begin{adjustwidth}{2em}{2em}\begin{ttfamily}\small}
        {\end{ttfamily}\end{adjustwidth}}

% One liner code/console command
\newcommand{\CODELINE}[1] {%
        \begin{CODE}%
        #1
        \end{CODE}%
}

% Text indented and italic
\newenvironment{IMPORTANT}
        {\begin{adjustwidth}{2em}{2em}\begin{itshape}}
        {\end{itshape}\end{adjustwidth}}


%%  Shortcuts

\newcommand{\COPYRIGHT} {%
        \small%
        \centerline{\rule{0.5\linewidth}{0.5pt}} %
        Written by H Fisher, Australian National University, 2024 \\
        Creative Commons BY-NC-SA license%
        \normalsize
}

% With smart editors interfering, safer not to type quotes directly
\newcommand{\DQ}[1] {%
        \textquotedblleft{}#1\textquotedblright{}%
}

\newcommand{\SQ}[1] {%
        \textquoteleft{}#1\textquoteright{}%
}

% Names of computers, hosts, programs
\newcommand{\NAME}[1] {%
        \texttt{\textbf{#1}}%
}

% For text, true dash to stop typographers complaining
\newcommand{\DSH} {\textendash{}\xspace}

% Bullet mark used outside list
\newcommand{\DOT} {\bullet\xspace}

% Introduce a new step in tutorial exercise
\newcounter{tutestep}
\newcommand{\STEP} {%
        \addtocounter{tutestep}{1}%
        \thetutestep.\xspace%
}


\begin{document}

\TITLE{TCP Client Server}


\MINOR{Outline}

In this tutorial you will

\DOT Study the differences between UDP and TCP program design

\DOT Study how data is transmitted and received over TCP

\DOT Study and change a simple protocol


This tutorial is very similar to the previous UDP client-server, so you should
know what to do. First make sure everything is
working: run the server in one window, the client in another. Whatever
you type in the client window will be sent to the server and echoed back. EOF
in the client window to stop, Control-C for the server.

For this tutorial you should have WireShark (or \texttt{tcpdump}) running
on the loopback interface, port 3310.


\SECTION{Program Design}

Open all the source code files (for your preferred language) in your editor
or IDE.

The TCP client and server both use a shared library for reading and writing
lines of text, \NAME{sockLine.py} or \NAME{SockLine.java}. Have a look at the
code.

Sending a line is easy: append a newline special character and then encode
into UTF-8. It is only a couple of lines, but using a library means we cannot
forget either step.

Reading a line is very different. The UDP programs received packets, and each
package was a single request or response. TCP hides the packets, so a TCP
socket behaves like a sequential file of bytes. The library code reads one
byte at a time, but this does not mean that every byte is a packet. WireShark
will show you what is actually being sent.

The TCP client uses the shared library, so the code for reading and writing
is shorter. One difference is that the client sends a final message on EOF:
this is a simple example of \NAME{protocol} design so the server can tell
whether a client deliberately ended the session or just crashed.

The TCP server code is structured very differently. A UDP server responds
to client packets and does not really need any state. A TCP server receives
connection requests from a client, and each connection is a \NAME{session}
dedicated to that client. The server now has two loops, one for client connections
and one for requests by a single client.


\SECTION{Data Over TCP}

In the UDP tutorial you could start a client with no server running. Try the
same with the TCP client.

Two UDP clients could run at the same time (in different terminal windows),
both sending requests to a single server. Try the same with the TCP server
and two clients. What happens?

\begin{IMPORTANT}
A real world server that only one person can use at a time would not be very
useful, but in this course we are studying computer networks, not Site
Reliability Engineering. Do \emph{not} use threads, timeouts, async\ldots
in your assignments without first asking if you really need to.
\end{IMPORTANT}

\STEP Modify the server \texttt{handleRequest} code so that it uses the special
\NAME{slowSend} in the socket line library instead of \texttt{writeLine}.
With a new server running, try sending requests of varying length from the
client. What does WireShark show? How many responses does the client receive?

Control-C the server and change the code back to using \texttt{writeLine}.
(If you don't, you will be waiting a very long time for the next step to
complete.)

\STEP The tutorial includes a text file \texttt{loremIpsum.txt} which is a
single 16K long line of randomly generated words. Use this as a client
request by running the client program with standard input from this file,
not the terminal.
\CODELINE{python tcpClient.py < loremIpsum.txt}
or
\CODELINE{java TcpClient  < loremIpsum.txt}

\begin{IMPORTANT}
This is an example of \SQ{fuzzing}, sending unexpected input to a program
and seeing what happens.
\end{IMPORTANT}

\STEP Using the \texttt{words.txt} file from the UDP tutorial, make sure
your client and server can handle non-ASCII text.


\SECTION{Protocol}

\STEP
In the \SQ{protocol} for this client server system a linefeed is added to
each line when transmitted, but it isn't being removed when received. This
makes it harder for either program to test if a line is  empty (\DQ{}) or
has a particular value, and inserts blank lines into the log output.

Modify the code for \texttt{readLine} in the shared library to remove the
linefeed (and any extra spaces) from the text once received and decoded.
(Hint for both Python and Java programmers: \texttt{trim}.)

\begin{IMPORTANT}
The general rule for layered network design is that any headers or trailers
added to encapsulate data at one end should be removed by the equivalent
layer at the other end. Here the LF is the protocol marker for a message
boundary.
\end{IMPORTANT}

\STEP
The client sends a BYE message when it shuts down. In the server, modify
\texttt{serverLoop}
so receiving this message exits the loop and closes the socket.

\STEP
Modify the server to send an extra empty line at the end of \texttt{handleRequest},
and modify the client \texttt{readReply} to keep reading lines until it receives
the empty line. (Which should not be logged.)

Run the new client and server and verify that everything still works with
simple one line messages and replies.

\begin{IMPORTANT}
When you add new features to a program, don't break what already works!
\end{IMPORTANT}

\STEP
Now that the client can loop until an empty line, modify the server so
that if the request is \DQ{it}, there is no reply, just the empty line.

\STEP
Modify the server so that if the request is \DQ{ni}, it sends back three
lines. \emph{Optional}: send back a random number of lines.


\SECTION{Experiments}

The TCP server can echo the input from other programs, not just the client.
Open a web browser and type \texttt{localhost:3310} as the URL.
Or try \NAME{telnet} or \NAME{nc} as the client. 

\emph{Optional}: Rewrite the server with a client session class, so that
each \texttt{serverLoop}
could be run as a thread to handle multiple client connections in parallel.
(You don't \emph{have} to actually make the server multithreaded, although
this is a good way to learn about threads if you haven't done so before.)


\COPYRIGHT

\end{document}
