

%   COMP3310 tutorial zero: basic programming
%   Written by H Fisher, ANU, 2024
%   Creative Commons CC-BY-4.0 license

%   LaTeX markup for XeTeX with a standard texlive distribution
%   Command to build:
%       xelatex 00-program.tex


%%%%    XeLaTeX common layout for COMP3310 tutorials

% Article class because tutorial instructions are short
\documentclass[a4paper,10pt,onecolumn,oneside,notitlepage]{extarticle}

% XeLaTex default input and output is UTF-8, so no need for character coding

\usepackage{changepage}
\usepackage{xspace}
\usepackage{graphicx}


%%  Fonts and characters

% With XeLaTeX can now use system fonts
\usepackage{fontspec}
\usepackage{unicode-math}

% Fonts! Everyone's favourite bike-shedding exercise.
% These fonts are distributed with texlive so should build everywhere.
\setmainfont{TeX Gyre Pagella}
\setsansfont{TeX Gyre Heros}
\setmonofont{TeX Gyre Cursor}
\setmathfont{TeX Gyre Pagella Math}


%%      Document structure

% Top of first page
\newcommand{\TITLE}[1] {%
        \begin{center}%
        \sffamily\bfseries\large%
        COMP3310/etc \DSH #1%
        \normalsize
        \end{center}%
}

% Section, subsection are sans-serif bold at normal text size,
% less spacing than usual between heading and paragraph.
\usepackage[sf,bf,tiny,raggedright,compact]{titlesec}

% Numbered subheading
\newcommand{\SECTION}[1] {%
        \section{#1}%
}

% Minor subheading
\newcommand{\MINOR}[1] {%
        \subsection*{#1}%
}


%%  Body text

% Try not to hyphenate
\sloppy

% Don't indent first line, put space between paragraphs
\setlength{\parindent}{0pt}
\setlength{\parskip}{1ex plus1ex minus0.2ex}

% Block of program code or console input and output, indented.
% Because code often has backslashes and curly brackets, always
% follow \begin{CODE} with \begin{verbatim} and finish with
% \end{verbatim} before \end{CODE}
% (No you can't put the begin/end verbatim in the environment.)
\newenvironment{CODE}
        {\begin{adjustwidth}{2em}{2em}\begin{ttfamily}\small}
        {\end{ttfamily}\end{adjustwidth}}

% One liner code/console command
\newcommand{\CODELINE}[1] {%
        \begin{CODE}%
        #1
        \end{CODE}%
}

% Text indented and italic
\newenvironment{IMPORTANT}
        {\begin{adjustwidth}{2em}{2em}\begin{itshape}}
        {\end{itshape}\end{adjustwidth}}


%%  Shortcuts

\newcommand{\COPYRIGHT} {%
        \small%
        \centerline{\rule{0.5\linewidth}{0.5pt}} %
        Written by H Fisher, Australian National University, 2024 \\
        Creative Commons BY-NC-SA license%
        \normalsize
}

% With smart editors interfering, safer not to type quotes directly
\newcommand{\DQ}[1] {%
        \textquotedblleft{}#1\textquotedblright{}%
}

\newcommand{\SQ}[1] {%
        \textquoteleft{}#1\textquoteright{}%
}

% Names of computers, hosts, programs
\newcommand{\NAME}[1] {%
        \texttt{\textbf{#1}}%
}

% For text, true dash to stop typographers complaining
\newcommand{\DSH} {\textendash{}\xspace}

% Bullet mark used outside list
\newcommand{\DOT} {\bullet\xspace}

% Introduce a new step in tutorial exercise
\newcounter{tutestep}
\newcommand{\STEP} {%
        \addtocounter{tutestep}{1}%
        \thetutestep.\xspace%
}


\begin{document}

\TITLE{Programming Warm Up}


\MINOR{Outline}

The goals of this tutorial are:

\DOT Make sure you have the necessary software for this course.

\DOT Make sure you can compile and run programs from the command line.

\DOT Introduce you to Inter Process Communication and the problems that can arise.

You don't have to finish this tutorial immediately. If you don't have the right
software, or don't understand how to do something, you can catch up. Ask your fellow
students or your tutor for help.


\SECTION{Software}

Tutorials are designed and tested on \NAME{Linux}, as used in the CSIT Lab workstations.
Usually everything will work on \NAME{MS Windows} or \NAME{MacOS X} (x86) without change,
or with minor modifications.

\NAME{Python} and \NAME{Java} are the \DQ{official} programming languages for this course.
Most tutorials are supplied in both these languages, and most students use one or the
other for their assignments. Most tutorials require only the standard built in libraries.

For Python any version from about 3.7 on will work. You won't need type annotations,
match-case statements, async code or other features from newer releases.

For Java use JDK 17, the long term release. You won't need virtual threads,
records, string templates, or other new features.

Tutorials are designed for a command line environment. On Linux and MacOS this is the
\NAME{Terminal} application. On MS Windows, use \NAME{PowerShell}, \emph{not} the command
prompt. We won't be writing shell scripts and usually the same commands work in all
three systems.

Why a CLI, Command Line Environment? This is not to make things deliberately hard,
or trying to bring back the computers of the previous century. Most networked computer
programs, and most computer network administration, does not use a GUI. This course is
about networks, not GUI programming.

You will need some way to edit your Python/Java programs. Any text editor will do:
\NAME{Sublime Text Editor} or \NAME{Atom} are good choices on any system, \NAME{BBEdit} on
Macs. You \emph{can} use an IDE such as \NAME{Eclipse}, \NAME{PyCharm}, or \NAME{Visual Studio}
if you prefer, but you \emph{must} be able to run your programs from the command line.
Assignments are submitted in source code form to your tutor, who probably won't have
the same IDE as you. If your programs can't run from the command line, they won't be
marked.

You need to know how to decompress / extract \NAME{ZIP} archives, as later on you will
be submitting your assignments in ZIP form. ZIP, \emph{not} 7z or RAR or tar.
(And don't put spaces or funny characters in file and folder names.)


\SECTION{Compile and run a program}

\begin{IMPORTANT}
Do, or do not. There is no try \DSH Yoda.

The tutorial exercises in this course are practical, not theoretical. If you think
you already know everything written here, do it anyway. Just to make sure.

If you can't do everything in one tutorial, that's OK too. The goal is to find
out what you need to learn before starting the actual tutorials and assignments.
\end{IMPORTANT}

The first program is \NAME{peasant.py} or \NAME{Peasant.java}. You can run from the
command line with
\begin{CODE}\begin{verbatim}
    python peasant.py
    ...
\end{verbatim}\end{CODE}
or
\begin{CODE}\begin{verbatim}
    javac Peasant.java
    java Peasant
    ...
\end{verbatim}\end{CODE}

\begin{IMPORTANT}
This tutorial shows the actual commands you should type. In future we will just ask you
to \DQ{compile and run the program} without this much detail.
\end{IMPORTANT}

This program expects you to type something and press Enter, which will then be printed
back to the terminal. It will keep doing so until End Of File, which in a terminal is
you typing \NAME{Control-D} or \NAME{Control-Z} depending on your system. If you don't
already know, find out which stops the program on your system.

The peasant program (Python or Java) has one optional command line argument.
In the source code this is handled in \NAME{processArgs} and you can see the effect
by running the program again. Some people prefer to study the code and then run the
program; other people prefer to run the program first and then look at the code.
There is no one right way.

\CODELINE{python peasant.py Dennis}
or
\CODELINE{java Peasant Dennis}

\begin{IMPORTANT}
Don't worry about the name of the program or the command line argument. This is an
old joke, because the tutorial was written by an old person.
\end{IMPORTANT}

What happens if you type more than one name on the command line?


\SECTION{Second program}

The second program is similar, in that it reads input and writes output until EOF,
but is not identical. Open the source code in your editor / IDE and run the program.

\begin{CODE}\begin{verbatim}
    python knight.py
\end{verbatim}\end{CODE}
or
\begin{CODE}\begin{verbatim}
    javac Knight.java
    java Knight
\end{verbatim}\end{CODE}

This program has new \NAME{chooseResponse} code that changes the output depending
on the input. One special input will cause the program to crash, another will
cause the program to loop forever. Read the code to find out what these strings
are. Run the program a few times and try both values. (If you don't already know,
you can stop a runaway program with \NAME{Control-C}.)


\SECTION{Inter Process Communication}

Send the output of the first program as the input to the second with command:

\CODELINE{python peasant.py | python knight.py}
or
\CODELINE{java Peasant | java Knight}

This is IPC, Inter Process Communication, one program communicating with another
although here without a computer network.

Type a few lines. Try the special input that crashes the knight program. What
happens? Do you need to type \NAME{Control-C} as well?

Reverse the program order:

\CODELINE{python knight.py | python peasant.py}
or
\CODELINE{java Knight | java Peasant}

Now what happens if you type the special crash input?


\SECTION{Coding}

Can the Java and Python programs interoperate? Use the IPC command, but instead
of both being Java or both being Python, try one of each.

Modify one program to stop on an empty line, ie pressing Enter without any text.

Put the knight program into an infinite loop. Instead of Control-C, use the
\NAME{ps} and \NAME{kill} commands (Linux and MacOS) or \NAME{Task Manager}
(MS Windows) to stop it. If you don't know how to do this, now is a good time
to learn.

\emph{Optional}: Modify the peasant program to join all the command line arguments
together (with a space in between each) to form a more complex name.


\COPYRIGHT

\end{document}
